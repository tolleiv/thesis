\newpage
\section{Ergebnistabellen zur Leistungsmessung}
\label{app:performance}


{
\renewcommand{\baselinestretch}{1.50} \normalsize
Ergebnisübersicht der in Abschnitt \ref{sec:performance} beschriebenen Leistungsevaluationen. Es wird jeweils Anzahl der parallelen Anfragen (Threads), der erzielte Durchsatz pro Sekunde (Requests/s), die Antwortzeiten und die Anzahl der durchgeführten Anfragen in Zeitraum von 60 Sekunden angegeben. Für die dabei gemessenen Antwortzeiten sind die minimalen, maximalen und mittleren Werte sowie das 90\%-Dezil in Millisekunden angegeben. Die Fehlerquote bezieht sich auf alle Verbindungsabbrüche oder  leere Apache Solr Ergebnisse.

\vfill
}

Tracker Leistungsmessung \\
{ \scriptsize
\centering
\csvreader[tabular=| r | r | r | r | r | r | r | r |,
    table head=\hline Threads & Requests/s & Min. Response & Max Response & Median Response & 90\%-Dezil & Fehlerquote & Samples \\\hline,
    late after line=\\\hline]%
{Results/aggregateTracker.csv}{ threads=\threads, count=\count, average=\average, median=\median, quantil=\quantil, min=\min, max=\max, err=\err, rate=\rate }%
{ \threads & \rate & \min & \max & \median & \quantil & \err & \count }%
}
\vfill\vfill\vfill
\newpage
%%%%%%%%%%%%%%%
Leistungsmessung einer einzelnen Apache Solr Instanz ohne Personalisierung. \\
{ \scriptsize
\centering
\csvreader[tabular=| r | r | r | r | r | r | r | r |,
    table head=\hline Threads & Requests/s & Min. Response & Max Response & Median Response & 90\%-Dezil& Fehlerquote & Samples \\\hline,
    late after line=\\\hline,
    filter=\equal{\personalized}{0} \and \equal{\cores}{1}]%
{Results/aggregateSearch.csv}{ threads=\threads, count=\count, average=\average, median=\median, quantil=\quantil, min=\min, max=\max, err=\err, rate=\rate, personalized=\personalized, cores=\cores }%
{ \threads & \rate & \min & \max & \median & \quantil & \err & \count }%
}

Leistungsmessung von zwei Apache Solr Instanzen ohne Personalisierung. \\
{ \scriptsize
\centering
\csvreader[tabular=| r | r | r | r | r | r | r | r |,
    table head=\hline Threads & Requests/s & Min. Response & Max Response & Median Response & 90\%-Dezil& Fehlerquote & Samples \\\hline,
    late after line=\\\hline,
    filter=\equal{\personalized}{0} \and \equal{\cores}{2}]%
{Results/aggregateSearch.csv}{ threads=\threads, count=\count, average=\average, median=\median, quantil=\quantil, min=\min, max=\max, err=\err, rate=\rate, personalized=\personalized, cores=\cores }%
{ \threads & \rate & \min & \max & \median & \quantil & \err & \count }%
}


Leistungsmessung von drei Apache Solr Instanzen ohne Personalisierung. \\
{ \scriptsize
\centering
\csvreader[tabular=| r | r | r | r | r | r | r | r |,
    table head=\hline Threads & Requests/s & Min. Response & Max Response & Median Response & 90\%-Dezil& Fehlerquote & Samples \\\hline,
    late after line=\\\hline,
    filter=\equal{\personalized}{0} \and \equal{\cores}{3}]%
{Results/aggregateSearch.csv}{ threads=\threads, count=\count, average=\average, median=\median, quantil=\quantil, min=\min, max=\max, err=\err, rate=\rate, personalized=\personalized, cores=\cores }%
{ \threads & \rate & \min & \max & \median & \quantil & \err & \count }%
\bigskip
}
\newpage
%%%%%%%%%%%%%%% 
Leistungsmessung einer einzelnen Apache Solr Instanz mit Webservice-basierter Personalisierung. \\
{ \scriptsize
\centering
\csvreader[tabular=| r | r | r | r | r | r | r | r |,
    table head=\hline Threads & Requests/s & Min. Response & Max Response & Median Response & 90\%-Dezil& Fehlerquote & Samples \\\hline,
    late after line=\\\hline,
    filter=\equal{\personalized}{1} \and \equal{\cores}{1}]%
{Results/aggregateSearch.csv}{ threads=\threads, count=\count, average=\average, median=\median, quantil=\quantil, min=\min, max=\max, err=\err, rate=\rate, personalized=\personalized, cores=\cores }%
{ \threads & \rate & \min & \max & \median & \quantil & \err & \count }%
}

Leistungsmessung von zwei Apache Solr Instanzen mit Webservice-basierter Personalisierung. \\
{ \scriptsize
\centering
\csvreader[tabular=| r | r | r | r | r | r | r | r |,
    table head=\hline Threads & Requests/s & Min. Response & Max Response & Median Response & 90\%-Dezil& Fehlerquote & Samples \\\hline,
    late after line=\\\hline,
    filter=\equal{\personalized}{1} \and \equal{\cores}{2}]%
{Results/aggregateSearch.csv}{ threads=\threads, count=\count, average=\average, median=\median, quantil=\quantil, min=\min, max=\max, err=\err, rate=\rate, personalized=\personalized, cores=\cores }%
{ \threads & \rate & \min & \max & \median & \quantil & \err & \count }%
}

Leistungsmessung von drei Apache Solr Instanzen mit Webservice-basierter Personalisierung. \\
{ \scriptsize
\centering
\csvreader[tabular=| r | r | r | r | r | r | r | r |,
    table head=\hline Threads & Requests/s & Min. Response & Max Response & Median Response & 90\%-Dezil& Fehlerquote & Samples \\\hline,
    late after line=\\\hline,
    filter=\equal{\personalized}{1} \and \equal{\cores}{3}]%
{Results/aggregateSearch.csv}{ threads=\threads, count=\count, average=\average, median=\median, quantil=\quantil, min=\min, max=\max, err=\err, rate=\rate, personalized=\personalized, cores=\cores }%
{ \threads & \rate & \min & \max & \median & \quantil & \err & \count }%
\bigskip
}
\newpage
%%%%%%%%%%%%%%% 
Leistungsmessung einer einzelnen Apache Solr Instanz mit Faktoren-basierter Personalisierung. \\
{ \scriptsize
\centering
\csvreader[tabular=| r | r | r | r | r | r | r | r |,
    table head=\hline Threads & Requests/s & Min. Response & Max Response & Median Response & 90\%-Dezil& Fehlerquote & Samples \\\hline,
    late after line=\\\hline,
    filter=\equal{\personalized}{2} \and \equal{\cores}{1}]%
{Results/aggregateSearch.csv}{ threads=\threads, count=\count, average=\average, median=\median, quantil=\quantil, min=\min, max=\max, err=\err, rate=\rate, personalized=\personalized, cores=\cores }%
{ \threads & \rate & \min & \max & \median & \quantil & \err & \count }%
}

Leistungsmessung von zwei Apache Solr Instanzen mit Faktoren-basierter Personalisierung. \\
{ \scriptsize
\centering
\csvreader[tabular=| r | r | r | r | r | r | r | r |,
    table head=\hline Threads & Requests/s & Min. Response & Max Response & Median Response & 90\%-Dezil& Fehlerquote & Samples \\\hline,
    late after line=\\\hline,
    filter=\equal{\personalized}{2} \and \equal{\cores}{2}]%
{Results/aggregateSearch.csv}{ threads=\threads, count=\count, average=\average, median=\median, quantil=\quantil, min=\min, max=\max, err=\err, rate=\rate, personalized=\personalized, cores=\cores }%
{ \threads & \rate & \min & \max & \median & \quantil & \err & \count }%
}

Leistungsmessung von drei Apache Solr Instanzen mit Faktoren-basierter Personalisierung. \\
{ \scriptsize
\centering
\csvreader[tabular=| r | r | r | r | r | r | r | r |,
    table head=\hline Threads & Requests/s & Min. Response & Max Response & Median Response & 90\%-Dezil& Fehlerquote & Samples \\\hline,
    late after line=\\\hline,
    filter=\equal{\personalized}{2} \and \equal{\cores}{3}]%
{Results/aggregateSearch.csv}{ threads=\threads, count=\count, average=\average, median=\median, quantil=\quantil, min=\min, max=\max, err=\err, rate=\rate, personalized=\personalized, cores=\cores }%
{ \threads & \rate & \min & \max & \median & \quantil & \err & \count }%
}

\newpage
\section{Referenzierte Repositories}
\label{app:repos}

Übersicht der Repositories für die im Rahmen dieser Arbeit erstellten Softwaremodule und Konfigurationsartefakte.

\begin{itemize}
\item \textbf{``Recommend''} - Hauptrepository mit allen Integrationskomponenten für beide Personalisierungslösungen und allen in Abschnitt \ref{sec:mahoutext} beschriebenen Ergänzungen für Apache Mahout. \\ https://gitorious.org/recommend/recommend
\item \textbf{``Tracker''} -  Umsetzung des in Abschnitt \ref{sec:tracker-impl} beschriebenen Protokollierungsdienstes. \\ https://gitorious.org/recommend/tracker
\item \textbf{``IMDB Index App``} - Beispielanwendung welche, basierend of NodeJS, beide Möglichkeiten der Personalisierung gegenüberstellt. \\ https://gitorious.org/recommend/imdb-index
\item \textbf{``Datenaufbereitung''}  - Scripte zur Datenaufbereitung für die Modellberechnung (vgl. Abschnitt \ref{sec:dataprep_real}). \\ https://gitorious.org/recommend/pig-preparation
\item \textbf{``Performance Tests''} - Apache Jmeter Konfigurationsdateien für die in Abschnitt \ref{sec:performance} durchgeführten Leistungsevaluationen. \\ https://gitorious.org/recommend/performance-tests
%\item recommend/pig-udfs
\item \textbf{``Apache Solr Konfiguration''} - Beispielkonfigurationen für die im Rahmen der Arbeit genutzten Apache Solr Instanzen. \\ https://gitorious.org/recommend/solr-config
\end{itemize}

