% !TEX encoding = UTF-8 Unicode
\documentclass[a4paper, ngerman, 12pt, onecolumn, oneside,bibliography=totoc,listof=totoc]{article}

\usepackage[utf8]{inputenc}
\usepackage[T1]{fontenc}
\usepackage{eurosym}
\usepackage[pdftex]{graphicx} % Einbindung von Grafiken (pdf, png, jpg)
\usepackage{float}            % bietet Option [H] für bombenfestes Verankern
\usepackage{wrapfig}
\usepackage{rotating}
\newcommand{\sturz}[1]{\rotatebox{90}{\parbox{20mm}{\raggedright #1}}} 
\usepackage[ngerman]{babel}   % Silbentrennung nach der neuen deutschen Rechtschreibung, z.B.: Sys-tem
%\usepackage{amstext}         % für Klartext via \text{} in Formeln
\usepackage{color}						% http://en.wikibooks.org/wiki/LaTeX/Colors
\usepackage[small,bf]{caption} % http://radio.wihome.net/wiki/?article=LaTeX:captions
\usepackage{sidecap}
\usepackage{amsmath}
\usepackage{amsfonts}					% wird with \mathbb gebraucht
\usepackage[safe]{tipa}
\usepackage{csvsimple}
\usepackage[titletoc]{appendix}
\usepackage{subfig}
	% ftp://ftp.tex.ac.uk/tex-archive/macros/latex/contrib/listings/listings.pdf
\usepackage{listings}
\renewcommand\lstlistlistingname{Listingverzeichnis}

\usepackage[pdftex,pagebackref=true]{hyperref}	% für "klickbare" Verlinkung im Dokument
\hypersetup{
    %bookmarks=true,        % show bookmarks bar?
    unicode=false,          % non-Latin characters in Acrobat’s bookmarks
    pdftoolbar=true,        % show Acrobat’s toolbar?
    pdfmenubar=true,        % show Acrobat’s menu?
    pdffitwindow=false,     % window fit to page when opened
    pdfstartview={FitH},    % fits the width of the page to the window
    pdftitle={Diplomarbeit - Skalierbare Item Recommendation in Big-Data und Suchindexen},    % title
    pdfauthor={Tolleiv Nietsch},     % author
    pdfsubject={},   % subject of the document
    pdfcreator={Tolleiv Nietsch},   % creator of the document
    pdfproducer={TU Chemnitz}, % producer of the document
    pdfkeywords={recommender,collective intelligence}, % list of keywords
    pdfnewwindow=true,      % links in new window
    colorlinks=true,       % false: boxed links; true: colored links
    linkcolor=black,          % color of internal links
    citecolor=black,        % color of links to bibliography
    filecolor=black,      % color of file links
    urlcolor=black           % color of external links
}

\usepackage[square]{natbib}
\bibliographystyle{natdin}
%\usepackage[natbib=true,citecounter=true]{biblatex}
%\addbibresource{Literatur.bib}
%\bibpunct{(}{)}{;}{a}{,}{,}

\usepackage{multicol}
\usepackage{multirow}

\usepackage[printonlyused]{acronym}
\renewcommand{\bflabel}[1]{\normalfont{\normalsize{#1}}\hfill}

\usepackage{parskip}          % zw. Absätzen: eine knappe Leerzeile statt hängender Einzüge
\usepackage{makeidx}          % Package zur Indexerstellung
\usepackage[toc,nonumberlist]{glossaries}
\usepackage[colorinlistoftodos,textsize=tiny]{todonotes}	% bringt den todo befehl mit - "disable" in die Optionen zum ausschalten
\usepackage{adjustbox}

\def\argmax{\operatorname*{arg\,max}}
\def\argmin{\operatorname*{arg\,min}}

\newenvironment{packed_enumerate}{
\begin{enumerate}
  \setlength{\itemsep}{1pt}
  \setlength{\parskip}{0pt}
  \setlength{\parsep}{0pt}}{\end{enumerate}
}

%% http://laclaro.wordpress.com/2011/07/30/svg-vektorgrafiken-in-latex-dokumente-einbinden/
\newcommand{\executeiffilenewer}[3]{%
\ifnum\pdfstrcmp{\pdffilemoddate{#1}}%
{\pdffilemoddate{#2}}>0%
{\immediate\write18{#3}}\fi%
}
\graphicspath{{./Abbildungen/}}
% includesvg[scale]{file} command (linux-version)
\newcommand{\includesvg}[2][1]{%
  \executeiffilenewer{#2.svg}{#2.pdf}{%
  /Applications/Inkscape.app/Contents/Resources/bin/inkscape -z -D --file="#2.svg" --export-pdf="#2.pdf" --export-latex}%
  \scalebox{#1}{\input{#2.pdf_tex}}%
}
%\graphicspath{{Abbildungen}}

\renewcommand{\figurename}{Abb.}
%%%%%%%%%%%%%%%%%%%%%%%%%%%%%%%%%%%%%%%%%%%%%%%%%%%%%%%%%%%%%%%%%%%%%%%%%%%%%%
%
% Größenanpassungen
%
\setlength{\unitlength}{1cm}
\setlength{\oddsidemargin}{0cm}
\setlength{\evensidemargin}{0cm}
\setlength{\textwidth}{16.1cm}
\setlength{\topmargin}{-1.2cm}
\setlength{\textheight}{23cm}
\columnsep 0.5cm
%
%%%%%%%%%%%%%%%%%%%%%%%%%%%%%%%%%%%%%%%%%%%%%%%%%%%%%%%%%%%%%%%%%%%%%%%%%%%%%%
\newcommand{\addtoindex}[1]{#1\index{#1}}
\newcommand{\nosubsubsection}[1] {{\textit{#1}}\newline}
\newcommand{\footnoteremember}[2]{
  \footnote{#2}
  \newcounter{#1}
  \setcounter{#1}{\value{footnote}}
} \newcommand{\footnoterecall}[1]{
  \footnotemark[\value{#1}]
} 

\sloppy                       % großzügiger Zeilenumbruch 
\makeindex
\makeglossaries

\hyphenation{Grund-elemente}	% definierte Silbentrennung
\hyphenation{Daten-menge}
\hyphenation{Werbe-banner}
\hyphenation{Boost-ing}

\begin{document}
\lstset{basicstyle=\linespread{1}\scriptsize\ttfamily, basewidth=0.51em,frame=lines,framerule=0.2pt,framesep=10pt}

\pagenumbering{roman}
\pagestyle{empty}
\include{Titelblatt}					% und eidesstattliche Erklärung

~
\vspace{5cm}
% siehe http://en.wikibooks.org/wiki/LaTeX/Document_Structure#Abstract

\begin{abstract}
%\begin{large}
%\textbf{Kurzfassung} \\ \\
%\end{large}

Die Integration von Suchtechnologien mit den Methoden des maschinellen Lernens bietet verschiedene Möglichkeiten um Suchergebnisse umfangreich zu personalisieren und so die Qualität der Suche für den Nutzer zu steigern. In dieser Arbeit werden die Grundlagen beider Technologien vorgestellt und zwei Möglichkeiten zur Integration, unter Berücksichtigung möglicher Herausforderungen bei der Skalierung, untersucht. Gegenübergestellt werden dabei elementbasierte Ähnlichkeitsmaße die mittels Webservice die Personalisierung einer Suche ermöglichen und faktorenbasierte Modelle welche die Personalisierungsberechnung direkt in der Suche integrieren. Verglichen werden die Leistungswerte und das Skalierungsverhalten, sowie die erzielbaren Qualität beider Lösungen. Die vorgestellte faktorenbasierte Personalisierung erwies sich dabei als qualitativ gleichwertige Alternative mit verbesserten Skalierungseigenschaften.

\end{abstract}
\newpage

\setcounter{page}{1}					% Seitenzahl zurück	
\tableofcontents					% Inhaltsverz.
\newpage

%\listoftodos \newpage

\pagenumbering{arabic}
\pagestyle{plain} 
\renewcommand{\baselinestretch}{1.50}\normalsize
\setcounter{page}{1}					% Seitenzahl zurück	

%%%%%%%%%%%%%%%%%%%%%%%%%%%%%%%%%%%%%%%%%%%%%%%%%%%%
%%%%%%%%%%%%%%%%%%%%

\newglossaryentry{AJAX}
{
  name=AJAX,
  description={ist ein Akronym for ``Asynchronous JavaScript and XML''. Es wird zur Datenübertragung zwischen Browser und Server genutzt. Die Übertragung kann auch sowohl synchron als auch asynchron erfolgen und die ausgetauschten Daten sind in der Regel per XML oder JSON serialisiert}
}
\newglossaryentry{REST}
{
  name=REST,
  description={ist ein Akronym für ``Representational State Transfer''. Es beschreibt die Nutzung von HTTP-basierten zustandslosen Diensten}
}
\newglossaryentry{ID}
{
  name=ID,
  description={bezeichnet eine Nummer die genutzt wird um Entitäten eindeutige zu referenzieren}
}
\newglossaryentry{SKU-ID}
{
  name=SKU-ID,
  description={ist ein Akronym für ``Stock keeping unit''. Es beschreibt eine ID, die die eindeutige Zuordnung zu einer bestimmten Bestandseinheit (Stückgut) und somit auch ihre Wiedererkennung ermöglicht}
}
\newglossaryentry{JSON}
{
  name=JSON,
  description={ist ein Akronym für ``JavaScript Object Notation''. Es ist ein für Menschen und Maschinen lesbares Dateiformat zum Datenaustausch und soll aus gültigen JavaScript bestehen}
}
\newglossaryentry{Recommender}
{
  name=Recommender,
  description={sind automatisierte Empfehlungstechnologien. Recommender filtern Informationen auf der Basis statistischer Daten, um damit Elemente (Produkte) aus einer Menge von Alternativen zu bestimmen, die entweder zum Nutzer oder zu einem anderen Element passen (vgl. Abschnitt \ref{sec:collaborativefiltering})}
}

\newglossaryentry{Servlet}
{
  name=Servlet,
  description={ beschreibt eine Java - Technologie um dynamische Webinhalte zu erstellen. Servlets laufen auf einem Webserver, um auf HTTP Anfragen dynamische Antworten zu erstellen}
}
\newglossaryentry{Tracking}
{
  name=Tracking,
  description={beschreibt das Erstellen eines Protokolls über das Nutzerverhalten auf einer Webseite}
}
\newglossaryentry{Webservice}
{
  name=Webservice,
  description={bezeichnet eine Technik zur Anwendungskopplung in heterogenen Systemen über standardisierte Protokolle. Der Datenaustausch erfolgt in der Regel über HTTP und die ausgetauschten Daten werden üblicherweise mittels XML oder JSON serialisiert}
}
\newglossaryentry{Reverse Proxy}
{
  name=Reverse Proxy,
  description={ist ein Netzwerkdienst über den der externe Zugriff auf Netz-interne Dienste gesteuert wird. Reverse Proxies werden u.a aus Sicherheitsgründen und zur Optimierung der Netzwerkleistung genutzt.}
 } 


\section{Einleitung}

Im reichhaltigen Angebot von Internetportalen und Online-Shops genügt es selten, im Kampf um Besucher, Informationen ansprechend darzustellen und Webseiten mittels integrierter Suchmaschine durchsuchbar zu machen. Nicht zuletzt wegen der herausragenden Stellung von Unternehmen wie Amazon oder Google sind Besucher an den Komfort von persönlichen Empfehlungen gewohnt und wechseln entnervt auf andere Angebote wenn Suchergebnisse nicht ihren Vorstellungen entsprechen \citep[Kap. 10]{hb,rs}. So wird die passende Personalisierung der auf Webseiten, Online-Shops oder Portalen verfügbaren Informationen zu einer zunehmenden Herausforderung für Unternehmen.

Stellt man sich dieser Herausforderung, steigt mit der Masse der Besucher auch der Umfang der zu verarbeitenden Daten. Mit jedem Klick eines Nutzers fallen neue Daten an, welche gefiltert und verarbeitet werden müssen. So kann ein plötzlicher Erfolg und ein unerhoffter Besucherstrom schnell zum Problem werden, wenn Suche und Personalisierung diesem nicht Stand halten. Neben dem Problem, möglichst gut personalisierte Inhalte zu präsentieren, muss deshalb auch die Skalierbarkeit betrachtet werden.

Danke quelloffener Software (OpenSource) und der aktiven Forschungs- und Entwicklungsgemeinde sind die Möglichkeiten solche Lösungen zu realisieren nicht nur großen Konzernen vorbehalten. Mit Suchlösungen, wie zum Beispiel Apache Lucene bzw. Apache Solr, lässt sich schnell eine gut in das Informations- oder Produkangebot integrierte Suchmaschine implementieren. Quelloffene Softwarebiblitotheken des maschinellen Lernens, wie zum Beispiel Apache Mahout, ermöglichen es, personalisierte Empfehlungen (Recommendations) zu berechnen. Durch die Integration beider Technologien ergibt sich so die Möglichkeit eine Suchlösung zu implementieren, die verfügbare Information optimal aufbereitet und entsprechend der persönlichen Präferenzen sortiert. \\ \\

\subsection{Zielsetzung}

In dieser Arbeit sollen die Möglichkeiten der Integration von Empfehlungsdiensten und Suchtechnologien untersucht werden. Neben der Vorstellung von Ansätzen und Algorithmen zur Empfehlungsbildung soll dafür vor allem das Suchergebnis-Boosting und die Kombination verschiedener Algorithmen zu diesem Zweck ausgearbeitet werden. Mit Hilfe der gewonnen Erkenntnisse soll eine Beispielanwendung implementiert werden, welche durch die Aufzeichnung des Nutzerverhaltens beim Gebrauch einer Webseite entsprechende Anpassungen bei der Generierung von Suchergebnissen ermöglicht.

Die entwickelte Beispielanwendung und die dargelegten Konzepte soll über den Rahmen der Arbeit hinaus in die Suchlösung ``Searchperience''\footnote{siehe: http://searchperience.me} der AOE GmbH integriert werden. Da diese auf den OpenSource Lösungen Apache Lucene und Apache Solr aufbaut, werden diese auch für die Umsetzung der Beispielanwendung vorausgesetzt. Des Weiteren sollen die Leistungsdaten der Searchperience Integration von QVC Italia\footnote{siehe: http://qvc.it} als Referenzwerte zum Leistungsvergleich genutzt werden.

\subsection{Gliederung der Arbeit}

Die Arbeit ist in die folgenden Abschnitte gegliedert. In Abschnitt \ref{sec:basics} werden zunächst die Grundlagen der genutzten Technologien erläutert und damit verbundene bekannte Schwierigkeiten aufgezeigt. Im darauf folgenden Abschnitt \ref{sec:architecture} werden Anforderungen und Struktur der Beispielanwendung erläutert. Die Beschreibung der zur Umsetzung genutzten Bestandteile erfolgt in Abschnitt \ref{sec:realization}. Die Bewertung der Technologien erfolgt in Abschnitt \ref{sec:evaluation}. Im abschließenden Abschnitt \ref{sec:results} wird die Arbeit zusammengefasst und mögliche weiterführende Themen werden aufgezeigt.

Ergänzend zur Arbeit werden in Anhang \ref{app:performance} die Evaluationsergebnisse aufgeschlüsselt und in Anhang \ref{app:repos} die zugehörigen Software Repositories aufgelistet.

%Hinweis das ``Elemente'' und ``Item'' oft als Synonym verwendet werden, ähnlich wie auch ``Rating'' und ``Bewertung'' gleichzusetzen sind.
%Goldberg92 -> Ursprung des Ausdrucks
%Ggf. direkt Terminologie ergänzen - gutes Beispiel Goldberg01 (Kap 3.1)

%\todo[color=red]{Hindernis}
%\todo[color=orange]{Wichtige Aufgabe}
%\todo[color=yellow]{Aufgabe abhängig von anderen}
%\todo[color=green]{Mögliche Ergänzung / weitere Anregung}
%\todo[color=white]{}
\newpage
\section{Grundlagen}
	Auf welchen Themen und Techniken baut die Arbeit auf.

\subsection{Recommendation Konzepte}
Die Auswahl von möglichst relevanten Empfehlungen für einen Nutzer kann auf sehr verschiedenen Wegen getroffen werden. Für die zahlreichen bekannten Techniken wird in der Literatur vorwiegend die folgende Gliederung genutzt \citep[Kap. 1]{hb} \citep{Burke:2002:HRS:586321.586352} \citep{rs}:

\begin{itemize}
\item \textit{Kollaboratives Filtern}, auch \textit{\acf{CF}}, gewinnt relevante Elemente aus dem Vergleich des Nutzerprofils mit anderen Profilen.
\item \textit{Community-basierte Empfehlungen}, bzw. \textit{Community-based filtering}, nutzen die Ähnlichkeit innerhalb von Gruppen, etwa in sozialen Netzwerken, um relevante Elemente zu finden.
\item \textit{Demographisch gestützte Empfehlungen} leiten sich von den Stereotypen, denen ein Nutzer zugeordnet wird, ab.
\item \textit{Inhaltsbasierte Empfehlungen} oder \textit{Content-based recommendations}, werden auf der Basis von, am Nutzerprofil gewichteten, Element-Eigenschaften getroffen.
\item \textit{Wissensbasierte Empfehlungen} bzw. \textit{Knowledge-based recommendations} werden durch zusätzliches domänenspezifisches Wissen generiert.
\item \textit{Utility-basierte Empfehlungen} bestimmen sich durch die Berechnung der Nützlichkeit der Elemente für den Nutzer mit Hilfe der sog. \textit{Utility Function}.
\item \textit{Hybride Systeme} kombinieren verschiedene Techniken um die Schwächen der einzelnen auszugleichen.
\end{itemize}

Die diesen Gruppen zugrunde liegenden Methoden werden in den nächsten Abschnitten näher erläutert. Dazu werden jeweils die zu erhebenden Daten, deren Verarbeitung und die Vor- und Nachteile der Methode beschrieben. % sowie mögliche Anwendungsgebiete beschrieben. 

%%%%%%%%%%%%%%%%%%%%%%%%%%%%%%%%%%%%%%%%%%%%%%%%%%%%%%%%%%%%%%%%%%%%%%%%%%%%%%
\subsubsection{Kollaboratives Filtern}
\label{sec:cf_overview}
Der Grundgedanke beim kollaborativen Filtern ist, dass Nutzer die in der Vergangenheit gleiche Interessen hatten diese auch in der Zukunft durch ähnliches Verhalten ausdrücken. So können Empfehlungen für einen Nutzer aus dem Verhalten ähnlicher Nutzer abgeleitet werden. Die Nutzerprofile bilden sich dabei ausschließlich aus Elementbewertungen (\textit{Ratings}), Eigenschaften der bewerteten Elemente fließen nicht ein.  Die Ähnlichkeit der Nutzer drückt sich entsprechend durch Gemeinsamkeiten in den Bewertungen aus. \citep[Kap. 2]{rs}

Aus den Profilen aller Nutzer ergibt sich eine sog. \textit{User-Item} Matrix, diese ermöglicht es ähnliche Nutzer oder auch ähnliche Elemente im System zu finden. Zur Auswertung dieser Matrix, bzw. zur Generierung von Empfehlungen mit Hilfe dieser Matrix existieren verschiedene Strategien welche in Abschnitt \ref{sec:filtermethods} näher beschrieben werden.

Die Erhebung der Ratings kann sowohl auf explizite Weise, etwa mit einer 5-Punkte-Likert-Skala, oder implizit, zum Beispiel durch die Aufzeichnung von Browsing-Verläufen, geschehen.

Ein wichtiger Vorteil des kollaborativen Filterns liegt darin, dass Empfehlungen unabhängig von Elementeigenschaften gebildet werden können. Dadurch können auch Elemente deren Inhalt nur schwer oder gar nicht gewonnen werden kann in die Empfehlung einbezogen werden. Die zahlreichen Forschungsarbeiten und die große Zahl der daraus hervorgegangenen Filterstrategien ist ebenfalls ein Vorteil.

Problematisch ist die Verwendung bei Systemen in denen der Nutzer (noch) kein oder nur ein sehr begrenztes Profil hat (\textit{cold start}). Zudem ist es nicht in jedem Fall sinnvoll alle Eigenschaften der Elemente ausser Acht zu lassen, da so ggf. problemspezifische Entscheidungskriterien unbeachtet bleiben.  \citep{hb,Burke:2002:HRS:586321.586352} %zum Beispiel der gewünschte Einsatzzweck beim Kauf einer Digitalkamera.%

\subsubsection{Community-basierte Empfehlungen}
Gemäß \citep{SinhaS01} haben Nutzer ein größeres Vertrauen in Empfehlungen wenn sie von Freunden ausgesprochen werden. Diesem Ansatz folgend werden in community-basierten Systemen Empfehlungen entsprechend der Präferenzen der Freunde eines Nutzers ausgesprochen. Das Nutzerprofil bildet sich daher aus einer Liste von Elementbewertungen und einer Liste von sozialen Verbindungen zu anderen Nutzern.

Da in Vergleichen mit reinen kollaborativen Systemen keine eindeutige Verbesserung der Empfehlungen nachgewiesen werden konnte, stellt das größere Vertrauen in die gebotenen Empfehlungen den wesentlichen Vorteil dieser Methode dar. Die gute Verbreitung und Verfügbarkeit der Daten über öffentliche Schnittstellen von  bestehenden sozialen Netzwerken, wie etwa Facebook oder LinkedIn, sind ebenfalls positiv. Die Abwägung zwischen der Aufrechterhaltung der Privatsphäre und dem dadurch resultierenden Verlust an Genauigkeit ist ein wichtiges Problem (vgl. \citep{machanavajjhala:accurate}). Auch das fehlende theoretische Fundament in anderen Bereichen, etwa beim Aufbau von Vertrauen und Misstrauen zwischen Nutzern, birgt mögliche Probleme bei der Umsetzung. \citep{hb_20}

%%%%%%%%%%%%%%%%%%%%%%%%%%%%%%%%%%%%%%%%%%%%%%%%%%%%%%%%%%%%%%%%%%%%%%%%%%%%%%
\subsubsection{Demographisch gestützte Empfehlungen}
Eine weitere Methode um ähnliche Nutzer zu finden ist die Gruppierung nach demographischen Eigenschaften. So können Gruppen zum Beispiel entsprechend des Alters, der Sprache oder des Geschlechts gebildet werden. Sie können allerdings auch mit Hilfe der Methoden des maschinellen Lernens aus bestehenden Transaktionsdaten gewonnen werden (vgl. \citep{Burke:2002:HRS:586321.586352}). Wie bei den vorangegangenen Methoden bildet sich auch hier das Nutzerprofil zunächst aus einer Liste von Elementbewertungen, ergänzt wird es durch die entsprechenden demographischen Eigenschaften. Die Empfehlungen für den einzelnen Nutzer ergeben sich aus seinen eigenen Präferenzen die endsprechend der Gruppenzugehörigkeit gewichtet werden.

Arbeiten zu reinen demographischen Systemen gibt es kaum. In vielen Fällen, wie etwa \citep{Vozalis:2007:USD:1243505.1243639} werden kollaborative Ansätze ergänzt  um eine Verbesserung der Empfehlungsergebnisse zu erzielen bzw. um die Probleme bei Empfehlungen für neue Nutzer zu verringern. \citep{Burke:2002:HRS:586321.586352}
% http://dx.doi.org/10.1016/j.ins.2007.02.036

%%%%%%%%%%%%%%%%%%%%%%%%%%%%%%%%%%%%%%%%%%%%%%%%%%%%%%%%%%%%%%%%%%%%%%%%%%%%%%
\subsubsection{Inhaltsbasierte Empfehlungen}
Bei der inhaltsbasierten Generierung von Empfehlungen werden die Element-Ratings eines Nutzers zur Erzeugung eines ``Interessenprofils'' genutzt. In diesem Profil drücken sich die Präferenzen des Nutzers für die inhaltlichen Eigenschaften der Elemente aus und so kann es direkt genutzt werden um ihm Elemente mit ähnlichen Eigenschaften zu empfehlen. Hat ein Nutzer also zum Beispiel ein `Harry Potter'' Buch positiv bewertet, so kann man leicht schlussfolgern dass auch andere Fantasy-Bücher empfohlen werden könnten.

Neben der automatischen Erstellung des Profils ist es auch möglich dieses explizit vom Nutzer zu erfragen. Abhängig vom Problemfeld kann dies schneller zu guten Empfehlungen führen und zur Steigerung des Vertrauens in die erzeugten Empfehlungen beitragen, vgl. \citep{hb_20}.

Zur Bestimmung ähnlicher Dokumente, bzw. zur Extraktion der relevanten Eigenschaften (\textit{Features}) werden abhängig vom Elementtyp verschiedene Methoden genutzt. Diese reichen von Entscheidungsbäumen über neuronale Netze bis hin zu Vektorraum-Verfahren (vgl. Abschnitt \ref{sec:filtermethods} und \citep[Kap. 3]{rs}). Die große Anzahl der dafür zur Verfügung stehenden Verfahren, die damit verbundenen Erfahrungen und das daraus abgeleitete Problembewusstsein ist einer der Vorteile. Wichtiger noch ist die Tatsache dass inhaltsbasierte Empfehlungen unabhängig von der Größe des Systems bzw. von das Anzahl der Nutzer generiert werden können. Ein weiterer Vorteil ist es dass für die so gewonnenen Empfehlungen auch leichter Erklärungen für den Nutzer generiert werden können, was wiederum ein wichtiger Faktor zur Steigerung des Vertrauens in die Qualität ist.

Schwierigkeiten bei der Erzeugung von Empfehlungen ergeben sich wenn die für den Nutzer relevanten Eigenschaften nicht direkt ``messbar'' vorliegen. Zum Beispiel des Ästhetik eines Produktes oder die Nutzbarkeit einer Webseite lassen sich nur sehr schwer erfassen, können aber beim Vergleich zweier Elemente wichtiger sein als textuelle Eigenschaften. Wie auch beim kollaborativen Filtern ist es bei dieser Methode sehr schwer gute Empfehlungen für Nutzer zu generieren, wenn diese kein oder nur ein unvollständiges Profil haben. Eine weitere Schwierigkeit ergibt sich daraus dass Empfehlungen nur aus dem "bevorzugten" Interessenbereich des Nutzers gewonnen werden, dies kann zu sehr ähnlichen und kaum ``überraschenden'' Empfehlungen führen und zu einem Problem was als \textit{more of the same} umschrieben wird.  \citep[Kap. 3]{rs} \citep{hb_03}

%Alternativ zur Ableitung des ``Interessenprofiles'' aus den Element-Ratings kann dieses auch direkt vom Nutzer erfragt werden. 

%%%%%%%%%%%%%%%%%%%%%%%%%%%%%%%%%%%%%%%%%%%%%%%%%%%%%%%%%%%%%%%%%%%%%%%%%%%%%%
\subsubsection{Wissensbasierte Empfehlungen}

Wenn die Frequenz mit der Nutzer ein Element brauchen oder konsumieren sehr gering ist, wie es etwa bei Hauskäufen der Fall ist, ergibt sich für die bisher beschriebenen Methoden das Problem dass nur selten umfangreiche Nutzerprofile zur Verfügung stehen oder die darin enthaltenen Informationen schlicht veraltet sind. Oft gibt es zudem in vielen Bereichen Expertenwissen bzw. domänenspezifisches Wissen welches zur Verbesserung von Empfehlungen bzw. zur Einschränkung der Kandidatenliste genutzt werden kann.

Um dieses vorhandene Wissen zur Generierung von Empfehlungen nutzbar zu machen, kann man es in eine Menge von Regeln überführen und mögliche Empfehlungen entsprechend der Regeln filtern. So kann man zum Beispiel aus der Information dass der Nutzer auf der Suche nach einer Wohnung für seine fünfköpfige Familie ist, leicht ableiten dass 40$m^{2}$ Wohnungen nicht empfehlenswert sind und dass solche mit zwei Bädern oder in einer ruhigeren Wohnlage empfohlen werden können.

Form und Inhalt des Nutzerprofils variieren hierbei in Abhängigkeit von der gewählten Wissens- bzw. Regelrepräsentation. Die Einbeziehung von Expertenwissen ermöglicht es auch übliche Standards einzubeziehen und es erleichtert die Vervollständigung des Nutzerprofils durch die Auswahl sinnvoller Fragen bei der Interaktion mit dem Nutzer. Auch in Fällen, in denen keine Vorschläge gefunden werden konnten, haben regelbasierte Systeme Vorteile. Die Information darüber dass keine Empfehlungen gefunden für eine Anfrage gefunden werden konnten, werden Nutzer schneller akzeptieren wenn das System zudem eine Reihe von Vorschlägen unterbreiten kann welche der Regeln ausgelassen werden könnten um neue Empfehlungen zu generieren. Nachteile ergeben sich wenn das Expertenwissen und die darauf basierenden Regeln nicht an neue Entwicklungen angepasst werden oder wenn für den Nutzer wichtige Features umbewertet bleiben. \citep[Kap. 4]{rs}

%%%%%%%%%%%%%%%%%%%%%%%%%%%%%%%%%%%%%%%%%%%%%%%%%%%%%%%%%%%%%%%%%%%%%%%%%%%%%%
\subsubsection{Utility-basierte Empfehlungen}

Ein zweiter Ansatz um domänenspezifisches Wissen zum Ausgangspunkt von Empfehlungen zu machen ergibt sich, indem man die ``Nützlichkeit'' eines Elements mit Hilfe einer nutzerspezifischen Funktion (\textit{Utility function}) berechnet. Dadurch kann zum Beispiel eine mögliche Toleranz des Nutzers gegenüber gewissen Produktmerkmalen direkt ins Verhältnis zur Dringlichkeit einer Bestellung gesetzt werden. Das Nutzerprofil ergibt sich dabei aus den Parametern der Funktion, welche i.d.R. explizit von Nutzer erfragt werden müssen.

Vor- und Nachteile sind ähnlich gelagert wie im vorangegangene Abschnitt. Vor allem der direkt Einfluss, den der Nutzer auf die Qualität der Ergebnisse hat, kann zur Steigerung das Vertrauens in die generierten Empfehlungen führen.  \citep[Kap. 1]{hb} \citep{Burke:2002:HRS:586321.586352, hb_20}






\subsection{Filtermodelle}
\label{sec:filtermethods}

Will man die in Abschnitt \ref{sec:cf_overview} beschriebenen kollaborativen Filtermethoden nutzen, stellt sich das Problem wie man die Ähnlichkeit von Nutzern oder Elementen bestimmen kann und wie man dann Empfehlungen für einen Nutzer erzeugt. Die dafür nötigen Modelle sollen in den folgenden Abschnitten näher erläutert werden.

Grundlage der im Folgenden beschriebenen Methoden ist eine \textit{User-Item} Matrix $R$ welche die Bewertung aller Nutzer $U$ für die Elemente (Produkte) $P$ enthält. Die Wahl des Wertebereichs hängt dabei von der Applikation ab. Ein Beispiel für eine solche Matrix wird in Tabelle \ref{tab:user-item-ratings} gezeigt.

% evt. Probleme dieser Darstellung im letzten Teil

\begin{table}
  \centering
  \begin{tabular}{ | l || c | c | c | c | c | c | c | }
    \hline
           & Item1 & Item2 & Item3 & Item4 & Item5 & Item6 & Item7 \\ \hline
User1 &    5.0 & 3.0      & 2.5     &            & & & \\				
User2 &    2.0 & 2.5      & 5.0     &  2.0    & & & \\
User3 & 2.5	& & &	4.0 &	 4.5	& &	5.0 \\
User4 & 5.0	& &	3.0	& 4.5 & &	4.0 &	 \\
User5 & 4.0	&3.0 &	2.0 &	4.0 &  3.5 & 4.0	& \\
    \hline
  \end{tabular}
  \caption{Beispiel-Matrix für User-Item Ratings}
  \label{tab:user-item-ratings}
\end{table}

\subsubsection{Ähnlichkeitsmaße}
\label{sec:similarities}

\paragraph{\addtoindex{Euklidische Distanz}} Die naheliegendste Form zur Bestimmung der Ähnlichkeit zwischen zwei Spalten oder zwei Zeilen der User-Item Matrix ist es, deren Abstand im $n$-dimensionalen euklidischen Raum, gem. Formel (\ref{form:eukildsim}) zu nutzen.
\begin{align}
\label{form:eukildsim}
dist(a,b) & = \sqrt{\sum_{i=1}^{n} (a_i - b_i)^2} \\
sim(a,b) & = \frac{1}{1+dist(a,b)} \label{form:disttosim}
\end{align}

Hierbei ist $n$ die Anzahl der Dimensionen und $a_i$ bzw. $b_i$ beziehen sich auf das  $i^{te}$ Attribut der Objekte, resp. die Ratings der Nutzer. Um den Distanzwert zu einem Maß der Ähnlichkeit mit einem Wertebereich von $1$ (starke Korrelation) bis $0$ (keine Korrelation) umzuformen, kann Formel (\ref{form:disttosim}) genutzt werden.

Aus der Verallgemeinerung dieser Berechnung, der sog. \textit{Lr-Norm} bzw. dem \textit{Minkowski Abstand}, ergeben sich weitere Abstandsmaße. Die sog. \textit{L1-Norm} (auch \textit{City-Block-} oder \textit{Manhattan-Distanz}) entspricht $r=1$, $r=2$ entspricht dem o.g. euklidische Abstand und $ r=\infty $ entspricht dem \textit{Tschebyscheff-Abstand}. \citep{hb_02}
\begin{align}
\label{form:minkowskisim}
dist(a,b) & = \sum_{i=1}^{n} (\left| a_i - b_i \right|^r)^\frac{1}{r}
\end{align}
% http://www.fernuni-hagen.de/imperia/md/content/ls_statistik/kurse/00883_lp2.pdf
% Anwendung finden die verschiedenen Abstandsmaße zum Beispiel in XXXXXX \todo{Anwendungsbeispiele raussuchen}

\paragraph{\addtoindex{Pearson-Korrelation}} Ein Problem bei der Berechnung mit der euklidischen Distanz ist, dass die Mittelwerte und Varianzen der Bewertungen einzelner Nutzer voneinander abweichen können obwohl diese vergleichbare ``Interessen'' haben (vgl. \citep[Kap. 2]{pci}). Dieser Mangel wird mit Hilfe der \textit{Pearson-Korrelation} (\ref{form:pearsonsim}) beseitigt.  Ihr Wertebereich reicht von $1$ (starke Korrelation) bis $-1$ (starke negative Korrelation). Vor Allem bei der Bestimmung von nutzerbasierten Ähnlichkeiten konnten mit ihr in vielen Fällen sehr gute Ergebnisse erzielt werden. Zudem existieren zahlreiche Erweiterungen, um zum Beispiel die Gewichtung von Übereinstimmungen bei der Bewertung von kontroversen Elementen stärker hervorzuheben. \citep[Kap. 2.1]{rs} \citep{hb_02}

\begin{align}
\label{form:pearsonsim}
sim(a,b) & = \frac{\sum_{p \in P} (r_{a,p}-\bar{r_a})(r_{b,p}-\bar{r_b})}{\sqrt{\sum_{p \in P} (r_{a,p}-\bar{r_a})^2 }\sqrt{\sum_{p \in P} (r_{b,p}-\bar{r_b})^2 }}
\end{align}

\paragraph{\addtoindex{Kosinus-Ähnlichkeit}}\label{sec:cossim} Ein weiterer Ansatz, der sich zum Standardmaß bei der Abbildung von Element- bzw. Item-Ähnlichkeit entwickelt hat, ist die \textit{Kosinus-Ähnlichkeit} (\ref{form:cossim}). Die Distanz zwischen zwei Vektoren entspricht dabei dem zwischen ihnen aufgespannten Winkel, entsprechend steigt die Ähnlichkeit von Vektoren wenn diese in die gleiche Richtung zeigen. 
\begin{align}
\label{form:cossim}
sim(a,b) & = \frac{a \cdot b}{\|a\| \|b\|}
\end{align}
Der Wertebereich des erzeugten Ähnlichkeitsmaßes liegt zwischen $1$ (starke Korrelation) und $0$ (keine Korrelation) wenn die genutzten Ausgangsvektoren nur positive Werte haben. Dies ist zum Beispiel der Fall bei den oft üblichen 5 Stern Rating-Skalen oder beim Vergleich von Textdokumenten anhand der Vorkommen einzelner Wörter. Das Maß reicht bis $-1$ für starke negative Korrelationen wenn auch negative Werte genutzt werden. \citep{rs}[Kap. 2.2]

\paragraph{\addtoindex{Jaccard-Koeffizient}} Liegen Ratings nur als binäre Werte vor, kann die Ähnlichkeit zweier Elemente durch das Verhältnis der Schnittmenge zur Vereinigungsmenge dieser definiert werden. Der Wertebereich des sog. \textit{Jaccard-Koeffizienten} (\ref{form:jaccardsim}) liegt ebenso zwischen $1$ und $0$. Verwendung findet er auch wenn die Werte wenig Informationen tragen, die Information ob ein Nutzer eine Bewertung abgegeben hat im Zentrum der Betrachtung steht oder durch die Rating-Werte Beziehungen zwischen Nutzern und Elementen (im Sinne eines Graphen) ausgedrückt werden. Erweitert wird der Jaccard-Koeffizent vom \textit{Tanimoto-} und vom \textit{\addtoindex{Dice-Koeffizienten}} (vgl. \citep{bogers09}). \citep[Kap. 3.1]{rs} \citep{pci}
\begin{align}
\label{form:jaccardsim}
sim(A,B) & = \frac{|A \cap B|}{|A \cup B|}
\end{align}

Welches der Distanzmaße für eine konkrete Anwendung genutzt werden sollte kann nicht pauschal beantwortet werden. Bei der Bestimmung von nutzerbasierten Ähnlichkeiten stellt in vielen Fällen die Peason-Korrelation einen guten Ausgangspunkt dar, beim Vergleich von Elementen ist die Kosinus Ähnlichkeit oft eine gute Wahl, aber in jedem Fall muss die Wahl eines Maßes immer mit einer entsprechenden Evaluation gegenüber anderen Maßen kontrolliert werden (vgl. Abschnitt \ref{sec:measures} u. \ref{sec:evaluation}).
\todo[color=green!40]{MAE dazu?}
% \paragraph{\addtoindex{Likelihood-Funktion}}

\subsubsection{Nachbarschaftsmodelle}\newpage

Die Information über die Ähnlichkeit zweier Nutzer kann mit Hilfe von Nachbarschaftsmodellen (\textit{Neighborhood Models}) zur Generierung von Empfehlungen genutzt werden. Um Empfehlungen für einen Nutzer aus den in Tabelle \ref{tab:user-item-ratings} gezeigten Ausgangsdaten abzuleiten, wird mit Hilfe der schon vorliegenden Ratings zunächst die Ähnlichkeit von dieses Nutzers zu anderen berechnet (siehe Tabelle \ref{tab:user-user-sim}). Um den möglichen Wert eines Elements für einen Nutzer aus diesen abzuleiten ($pred(u,p)$), werden die Ratings anderer Nutzer für dieses Element entsprechend der Ähnlichkeit zwischen den Nutzern (siehe Formel  \ref{form:calcpred})) aufsummiert und normiert. Hierbei muss zudem ein weiterer Unterschied zwischen einzelnen Nutzern in Betracht gezogen werden. Auch wenn Nutzer generell ähnliche Interessen oder Meinungen haben, so kann es durchaus sein, dass der Mittelwert der Ratings dieser Nutzer sehr verschieden ist. \todo{Quelle dazu - Töscher 2008 }. Diese Gewichtung am Rating-Mittelwert $\overline{r_u}$ der Nutzer wird aus diesem Grund in der erweiterten Formel (\ref{form:calcmeanpred}) beachtet. \citep{rs}
\begin{table}
  \centering
  \begin{tabular}{ | l || c | c | c | c | c | c | c | }
    \hline
           & User2 & User3 & User4 & User5 \\ \hline
User1 &    0.203 &	0.286 &	0.667 &	0.472 \\	
    \hline
  \end{tabular}
  \caption{Aus Tabelle \ref{tab:user-item-ratings} mit der euklidischen Distanz abgeleiteter Ähnlichkeitsvektor für User1}
  \label{tab:user-user-sim}
\end{table}
\begin{align}
pred(u,p) & = & \frac{ \sum_{b \in U} sim(u, b)*r_{b,p}}{ \sum_{b \in U} sim(u,b) } \label{form:calcpred} \\
pred(u, p) & = & \overline{r_u} + \frac{ \sum_{b \in U} sim(u, b)*(r_{b,p}-\overline{r_b}) } { \sum_{b \in U} sim(u,b) } \label{form:calcmeanpred}
\end{align}

%Nachbarschaftsformen
\citep{hb_04}
%User vs. Item

%Vorteil 

%Nachteile


\subsubsection{SlopeOne}

\citep[S 41]{rs}

\subsubsection{Matrixfaktorisierung}

% SVD - Abbildung HB S. 46
\citep{Koren:2009:MFT:1608565.1608614}

\subsubsection{Graphen Modelle}\newpage


\subsection{Schwierigkeiten von Recommendern}

- Allgemeine Recommendation Challenges und Ansätze (Sparse, Grey Sheep, "rich-gets-richer"...)
- Latenzzeit für das lernen (Nutzer hat grad was angeschaut)
- Lösung wie dus gemacht hast (Real time learning)
- Item based ohne Nutzerid
- Zeitliche Aspekte (Weihnachtsgeschäft etc) - (hattest du ja ausgeführt)


\subsection{Suchindexe}
\subsubsection{Dokumentenrepräsentation}
\subsubsection{Relevanzberechnung}\newpage

% http://lucene.apache.org/core/3_6_1/api/all/org/apache/lucene/search/Similarity.html#formula_termBoost

\subsection{Skalierungsstrategien}


\citep{linden03} Amazon.com recommendations: item-to-item collaborative filtering

 - Recommendation Datenmodelle die "schnell" sind
- die meisten arbeiten mit "In-Memory" Datenmodellen (mit Optimierte Speicherung für die Ausführung)
        - Wo sind die Grenzen (user,item anzahl ?)
        - Lösungen: Precomputing similarities? 
  - Lösungen: Cluster-based recommendation
- Wann macht distributed recommendation sinn
  - Wann stoßen memory based ans limit, welche algorithmen gibt es
  - Ist Offline recommendation dafür ein Use-Case (Calculation recommendations and send via mail)?
  - Hadoop vs Sharding
  - Das lernen der Datenmodelle muss auch skalieren
- Hadopp etc...

\citep{Vidal:2005:PDR:2137725.2137737}
\citep{Toscher:2008:INA:1722149.1722153}
\citep{linden03}

\subsubsection{Online- / Offline-Recommendation}
\subsubsection{MapReduce basierte Algorithmen}
\citep{mapred008}
% RS Seite  48
% Sharding (der Solr indexe)

\subsection{Qualitätsmaße}\label{sec:measures}

	\begin{itemize}
	\item Successful session \citep{hb_18,Smyth05alive-user}
	\item Precision /Recall /F1, \citep{rs}[Kap. 7]
	\item User- und Itemcoverage \citep{rs}[S. 183]
	\item Conversion / Click-through rates
	\end{itemize}

\subsubsection{Mittlere Abweichung}\newpage
% MAE, RMS
\subsubsection{Trefferquote und Genauigkeit}\newpage
% Precision, Recall, F1, Fall-Out
\subsubsection{Empirische Messung}\newpage
% Successful session

%\subsubsection{Normalisierter DCG}
%\subsubsection{A/B Testing}
\newpage

\section{Entwurf}

\subsection{Anforderungen}\newpage.\newpage

Potentielle Use-Cases:

A) Personalisierte Suche:
  Wenn man nach allgemeinen Begriffen wie "Geschenk" "Wein" oder "Kleid" sucht, bekommt man bereits durch andere Nutzer gelernte passende Empfehlungen in der Suche höher angezeigt.
  (Nutzer schaut sich Rotweine an und bekommt bei Suche nach Wein passende Rotweine angezeigt. Im Vergleich dazu ein Nutzer der sich Weißweine angeschaut hat)

B) Personalisiertes-Recommendation Widget:
 - In der rechten Spalte werden (passend zu den letzten besuchten Items oder durchgeführten Suchen) persönliche Empfehlungen gegeben


C) Context-Recommendation Widget:
1) Alla "Nutzer die x gekauft haben haben auch y gekauft" oder simple "Find most simelar items to an item"
Hier macht die Kombination von Suchindex und Recommendation auch Sinn weil:
- Der Suchindex alle zur Ausgabe relevanten Daten eines Dokumentes hat
- (Vorberechnung?)
- Es einfach möglich ist die Items über die die Recommendation gemacht werden sollen über inhaltliche (regelbasierte) Suchqueries weiter einzuschränken oder zu boosten (e.g. Items die höheren Preis haben und in gleicher Kategorie sind höher boosten)

2) Crosselling Widget im Warenkorb (Empfehlungen zu den Items im Warenkorb)



\subsection{Systemarchitektur}\newpage.\newpage



-Wenn man von großen Mengen (großer Raum) von Items ausgeht stellt sich das Problem, das die Teilmenge der Items die der Recommender und die Suche zu einer "Anfrage" zurück geben können disjunkt sein oder nur eine unerhebliche Schnittmenge haben. Teilprobleme wären
   - "Boosting in der Suchmenge": Recommender bekommt Solr-Menge und macht nur darüber Recommendations (wenn überhaupt möglich - vielleicht kommt man ja in die erste Schleife der user-based algorithmen "i that u has no preference for yet")
   - "Boosting in der Recommender Menge": Recommender bestimmt Menge (OR query + Optionales query) und Solr boosted nur noch darin. (Use-Case widget)
   -  "Selective Recommendation": Recommender wählt eine zum Query passendes Datenmodell / Datengrundlage aus, die möglichst viele potentielle Schnittmengen mit dem Query hat
       - Hier könnte man Jobs haben die regelmäßig die User-Item Datensätze nur für Items lernen die von der Suche zu einem bestimmtem Query zurückkommen (e.g. die Top-Querys)
   - "Anreicherungsansatz" Der Recommender könnte zu Ergebnissen in der Suchmenge weitere Ergebnisse einstreuen (flgl "More like these" handler).
   - "Precomputation and Delegation to Solr"
- Da find ich auch Interessant. Man kann sicherlich durch vorclustern ähnlicher Items oder ähnliche Items pro Usercluster alles direkt in einer Abfrage abwickelt  (Durch anreichern der Daten im Index )

- Recommender empfehlen nur neue Items, man will aber ggf auch das bereits geklickte Items in der Suche höher gewichtet werden (so macht es Google ja auch)
- Recommendation Algorithmen und Datengrundlage sind sehr verschieden. Wie kann man verschiedene Implementierungen nutzen/ansprechen/auswählen.


\subsection{Datenerhebung}\newpage


\section{Realisation}\newpage.\newpage\newpage.\newpage
	
	Verfeinerung des Entwurfs, Schilderung bei der Umsetzung aus dem vorangegangenen Kapitel

% Vergleich aus http://girlincomputerscience.blogspot.de/2012/11/open-source-recommendation-systems.html
	
\subsection{Apache Mahout}
\subsubsection{Bestandteile}
\subsubsection{Datenaufbereitung}
\subsubsection{Skalierung}
\subsection{Apache Solr}

\subsubsection{Indexierung}
\subsubsection{Score-Anpassung}
% http://lucene.apache.org/core/3_6_1/api/all/org/apache/lucene/search/Similarity.html#formula_termBoost

\lstinputlisting[caption=solrconfig.xml Anpassungen]{Listings/solrconfig.xml}
\lstinputlisting[caption=schema.xml Anpassungen]{Listings/schema.xml}
\lstinputlisting[caption=Solr-Anfrage um Vektor-Distanz einzubeziehen]{Listings/solrqueries.txt}


\subsection{Searchperience Integration}
\section{Evaluation}\label{sec:evaluation}\newpage.\newpage\newpage.\newpage\newpage.\newpage\newpage.\newpage\newpage.\newpage

	Wie wird gemessen, welche Ergebnisse waren zu erwarten, was wurde erreicht. Warum gibt es Abweichungen, welche Probleme enthält die Messmethode.
	
\subsection{Ergebnisse}
\subsection{Diskussion}



\section{Zusammenfassung}\label{sec:results}

%\textit{Abriss der Arbeit, was wurde erreicht bzw. gelernt. An welcher Stellen kann weitergearbeitet werden.}

\subsection{Fazit}

In dieser Arbeit wurden die Möglichkeiten zur Integration von Suchindexen und Empfehlungsdiensten untersucht. Unter Berücksichtigung der Skalierbarkeit wurden zu diesem Zweck zwei mögliche Lösungswege betrachtet. Diese ergänzen die bekannten Personalisierungsmethoden (vgl. Abschnitt \ref{sec:personalresultstheorie}) durch die direktere Integration der Methoden des kollaborativen Filterns mit Suchindexen. Zum Einen wurden die Ergebnisse eines Empfehlungsdienstes genutzt um Suchergebnisse zu personalisieren. Im zweiten Ansatz wurden Möglichkeiten zur direkten Empfehlungsbildung im Suchindex untersucht, um beide Dienste vollständig integriert nutzen zu können.

In der im Rahmen der Arbeit entwickelten Beispielapplikation wurden beide Personalisierungslösungen hinsichtlich ihrer Qualität und Leistungsfähigkeit evaluiert. Dabei hat sich gezeigt, dass die bekannten und oft untersuchten Algorithmen des kollaborativen Filterns (vgl. Abschnitt \ref{sec:neighborhoods}) bezüglich der Qualität bessere Ergebnisse liefern als die auf Matrixfaktorisierung basierende Personalisierungslösung. Durch die direktere Integration in Apache Solr liefern diese allerdings erheblich bessere Leistungswerte.

Umgesetzt wurde die Beispielapplikation auf Basis der quelloffenen Software Apache Solr und Apache Mahout.
\newpage
\subsection{Ausblick}

Obwohl im Rahmen dieser Arbeit zwei Möglichkeiten zur Personalisierung von Suchergebnissen erfolgreich implementiert wurden, verbleiben einige Fragen zur weiteren Betrachtung. Die in der Evaluation festgestellten Skalierungsprobleme von Apache Solr bei niedrigem Parallelisierungsgraden, mögliche Probleme der Zwischenspeichereffizienz bei der Personalisierung und die Optimierung der Parameter bei der Modellberechnung bedürfen weiteren Untersuchungen vor einem praktischen Einsatz. Bei der Verwendung der Personalisierung mittels Webservice gilt es, das Problem der disjunkten Kanidatenlisten genauer zu untersuchen.

Im Umgang mit dem Besucher bzw. Kunden einer Webseite sind im Zusammenhang mit den vorgestellten Personalisierungskonzepten Aspekte zu möglichen Einflüssen auf die Verkaufsdiversität (vgl. \citep{Fleder09}) und Möglichkeiten zur Integration von Kritikmechanismen (vgl. \citep{hb_13}) umbetrachtet geblieben.

Bei den eingesetzten Algorithmen ergeben sich ebenfalls Anknüpfungspunkte. Im Bereich der Matrixfaktoriersung existieren zahlreiche Erweiterungen zur Integration von implizitem und explizitem Feedback \citep{Joachims05} und zur Beachtung des temporalen Kontextes \citep{Boughareb11}. Daneben stellt die Kombination mehrerer Empfehlungsalgorithmen in einer Lösung, wie sie zum Beispiel bei der Netflix-Price-Competition notwendig war \citep{netflix2012_2}, einen weiteren wichtigen Aspekt zur Steigerung der Empfehlungsqualität dar. Wie \citep{Forbes11} zeigt, gilt dies ebenso  für die Kombination verschiedener Empfehlungskonzepte. 

Mit diesen zahlreichen Aspekten sollte die Integration eines Empfehlungssystems sicherlich als stetiger Prozess betrachtete werden, der durch neue Erkenntnisse und Algorithmen aus der Forschung vorangetrieben werden kann.



%\begin{itemize}
%\item Ggf. ``Lesezeit'' als Maß einbringen bzw. explizite und implizites Feedback (siehe Joachims05 - Abschnitt 2)
%\item Annahme das sich die Präferenzen der Nutzer über die Zeit nicht änder ist ggf. falsch <-> Temporaler Kontext aus Boughareb11 
%\item Auswirkungen auf die Verkaufsdiversität Fleder09
%\item Ausblick ``Rollout'' \citep{netflix2012_2}
%\item Ausblick ``Similarity Search in High Dimensions via Hashing'' (LSH) als mögliche Erweiterung
%\item Ausblick ``Content boosted matrix factorization'' with in Forbes11
%\item Ausblick andere Suchlösungen - Elastik Search etc..
%\end{itemize}

%%%%%%%%%%%%%%%%%%%%%%%%%%%%%%%%%%%%%%%%%%%%%%%%%%%%%%%%%%%%%%%%%%%%%%%%

\newpage\printglossary[title=Glossar,toctitle=Glossar] 

\phantomsection
\addcontentsline{toc}{section}{Abkürzungsverzeichnis}
\section*{Abkürzungsverzeichnis}
\begin{acronym}[XXXXXXX]
	\setlength{\itemsep --- }{-\parsep}
  \setlength{\itemsep}{1pt}
  \setlength{\parskip}{0pt}
  \setlength{\parsep}{0pt}
	\acro{CF}{Collaborative Filtering}
	\acro{IR}{Information Retrieval}
	\acro{SVD}{Singular Value Decomposition}
\end{acronym}

\newpage
\phantomsection
\addcontentsline{toc}{section}{Abbildungsverzeichnis}
\listoffigures

\phantomsection
\addcontentsline{toc}{section}{\lstlistlistingname}
\lstlistoflistings

\interlinepenalty=10000
%\nocite{*}							% auch die nicht verwendeten bibtex-Einträge einblenden
\cleardoublepage						% damit die TOC auf die richtige seite verweist
\phantomsection						% für hyperref um auf die richtige section zu verlinken
\addcontentsline{toc}{section}{Literatur}
%\renewcommand\refname{I Literatur}

{
\renewcommand{\baselinestretch}{1.0}\scriptsize
\bibliography{Literatur}}

\interlinepenalty=100
\renewcommand{\baselinestretch}{1.0}\small
%\appendix
%\phantomsection
%\addcontentsline{toc}{section}{Anhang}
\begin{appendices}
\newpage
\section{Ergebnistabellen zur Leistungsmessung}
\label{app:performance}


{
\renewcommand{\baselinestretch}{1.50} \normalsize
Ergebnisübersicht der in Abschnitt \ref{sec:performance} beschriebenen Leistungsevaluationen. Es wird jeweils Anzahl der parallelen Anfragen (Threads), der erzielte Durchsatz pro Sekunde (Requests/s), die Antwortzeiten und die Anzahl der durchgeführten Anfragen in Zeitraum von 60 Sekunden angegeben. Für die dabei gemessenen Antwortzeiten sind die minimalen, maximalen und mittleren Werte sowie das 90\%-Dezil in Millisekunden angegeben. Die Fehlerquote bezieht sich auf alle Verbindungsabbrüche oder  leere Apache Solr Ergebnisse.

\vfill
}

Tracker Leistungsmessung \\
{ \scriptsize
\centering
\csvreader[tabular=| r | r | r | r | r | r | r | r |,
    table head=\hline Threads & Requests/s & Min. Response & Max Response & Median Response & 90\%-Dezil & Fehlerquote & Samples \\\hline,
    late after line=\\\hline]%
{Results/aggregateTracker.csv}{ threads=\threads, count=\count, average=\average, median=\median, quantil=\quantil, min=\min, max=\max, err=\err, rate=\rate }%
{ \threads & \rate & \min & \max & \median & \quantil & \err & \count }%
}
\vfill\vfill\vfill
\newpage
%%%%%%%%%%%%%%%
Leistungsmessung einer einzelnen Apache Solr Instanz ohne Personalisierung. \\
{ \scriptsize
\centering
\csvreader[tabular=| r | r | r | r | r | r | r | r |,
    table head=\hline Threads & Requests/s & Min. Response & Max Response & Median Response & 90\%-Dezil& Fehlerquote & Samples \\\hline,
    late after line=\\\hline,
    filter=\equal{\personalized}{0} \and \equal{\cores}{1}]%
{Results/aggregateSearch.csv}{ threads=\threads, count=\count, average=\average, median=\median, quantil=\quantil, min=\min, max=\max, err=\err, rate=\rate, personalized=\personalized, cores=\cores }%
{ \threads & \rate & \min & \max & \median & \quantil & \err & \count }%
}

Leistungsmessung von zwei Apache Solr Instanzen ohne Personalisierung. \\
{ \scriptsize
\centering
\csvreader[tabular=| r | r | r | r | r | r | r | r |,
    table head=\hline Threads & Requests/s & Min. Response & Max Response & Median Response & 90\%-Dezil& Fehlerquote & Samples \\\hline,
    late after line=\\\hline,
    filter=\equal{\personalized}{0} \and \equal{\cores}{2}]%
{Results/aggregateSearch.csv}{ threads=\threads, count=\count, average=\average, median=\median, quantil=\quantil, min=\min, max=\max, err=\err, rate=\rate, personalized=\personalized, cores=\cores }%
{ \threads & \rate & \min & \max & \median & \quantil & \err & \count }%
}


Leistungsmessung von drei Apache Solr Instanzen ohne Personalisierung. \\
{ \scriptsize
\centering
\csvreader[tabular=| r | r | r | r | r | r | r | r |,
    table head=\hline Threads & Requests/s & Min. Response & Max Response & Median Response & 90\%-Dezil& Fehlerquote & Samples \\\hline,
    late after line=\\\hline,
    filter=\equal{\personalized}{0} \and \equal{\cores}{3}]%
{Results/aggregateSearch.csv}{ threads=\threads, count=\count, average=\average, median=\median, quantil=\quantil, min=\min, max=\max, err=\err, rate=\rate, personalized=\personalized, cores=\cores }%
{ \threads & \rate & \min & \max & \median & \quantil & \err & \count }%
\bigskip
}
\newpage
%%%%%%%%%%%%%%% 
Leistungsmessung einer einzelnen Apache Solr Instanz mit Webservice-basierter Personalisierung. \\
{ \scriptsize
\centering
\csvreader[tabular=| r | r | r | r | r | r | r | r |,
    table head=\hline Threads & Requests/s & Min. Response & Max Response & Median Response & 90\%-Dezil& Fehlerquote & Samples \\\hline,
    late after line=\\\hline,
    filter=\equal{\personalized}{1} \and \equal{\cores}{1}]%
{Results/aggregateSearch.csv}{ threads=\threads, count=\count, average=\average, median=\median, quantil=\quantil, min=\min, max=\max, err=\err, rate=\rate, personalized=\personalized, cores=\cores }%
{ \threads & \rate & \min & \max & \median & \quantil & \err & \count }%
}

Leistungsmessung von zwei Apache Solr Instanzen mit Webservice-basierter Personalisierung. \\
{ \scriptsize
\centering
\csvreader[tabular=| r | r | r | r | r | r | r | r |,
    table head=\hline Threads & Requests/s & Min. Response & Max Response & Median Response & 90\%-Dezil& Fehlerquote & Samples \\\hline,
    late after line=\\\hline,
    filter=\equal{\personalized}{1} \and \equal{\cores}{2}]%
{Results/aggregateSearch.csv}{ threads=\threads, count=\count, average=\average, median=\median, quantil=\quantil, min=\min, max=\max, err=\err, rate=\rate, personalized=\personalized, cores=\cores }%
{ \threads & \rate & \min & \max & \median & \quantil & \err & \count }%
}

Leistungsmessung von drei Apache Solr Instanzen mit Webservice-basierter Personalisierung. \\
{ \scriptsize
\centering
\csvreader[tabular=| r | r | r | r | r | r | r | r |,
    table head=\hline Threads & Requests/s & Min. Response & Max Response & Median Response & 90\%-Dezil& Fehlerquote & Samples \\\hline,
    late after line=\\\hline,
    filter=\equal{\personalized}{1} \and \equal{\cores}{3}]%
{Results/aggregateSearch.csv}{ threads=\threads, count=\count, average=\average, median=\median, quantil=\quantil, min=\min, max=\max, err=\err, rate=\rate, personalized=\personalized, cores=\cores }%
{ \threads & \rate & \min & \max & \median & \quantil & \err & \count }%
\bigskip
}
\newpage
%%%%%%%%%%%%%%% 
Leistungsmessung einer einzelnen Apache Solr Instanz mit Faktoren-basierter Personalisierung. \\
{ \scriptsize
\centering
\csvreader[tabular=| r | r | r | r | r | r | r | r |,
    table head=\hline Threads & Requests/s & Min. Response & Max Response & Median Response & 90\%-Dezil& Fehlerquote & Samples \\\hline,
    late after line=\\\hline,
    filter=\equal{\personalized}{2} \and \equal{\cores}{1}]%
{Results/aggregateSearch.csv}{ threads=\threads, count=\count, average=\average, median=\median, quantil=\quantil, min=\min, max=\max, err=\err, rate=\rate, personalized=\personalized, cores=\cores }%
{ \threads & \rate & \min & \max & \median & \quantil & \err & \count }%
}

Leistungsmessung von zwei Apache Solr Instanzen mit Faktoren-basierter Personalisierung. \\
{ \scriptsize
\centering
\csvreader[tabular=| r | r | r | r | r | r | r | r |,
    table head=\hline Threads & Requests/s & Min. Response & Max Response & Median Response & 90\%-Dezil& Fehlerquote & Samples \\\hline,
    late after line=\\\hline,
    filter=\equal{\personalized}{2} \and \equal{\cores}{2}]%
{Results/aggregateSearch.csv}{ threads=\threads, count=\count, average=\average, median=\median, quantil=\quantil, min=\min, max=\max, err=\err, rate=\rate, personalized=\personalized, cores=\cores }%
{ \threads & \rate & \min & \max & \median & \quantil & \err & \count }%
}

Leistungsmessung von drei Apache Solr Instanzen mit Faktoren-basierter Personalisierung. \\
{ \scriptsize
\centering
\csvreader[tabular=| r | r | r | r | r | r | r | r |,
    table head=\hline Threads & Requests/s & Min. Response & Max Response & Median Response & 90\%-Dezil& Fehlerquote & Samples \\\hline,
    late after line=\\\hline,
    filter=\equal{\personalized}{2} \and \equal{\cores}{3}]%
{Results/aggregateSearch.csv}{ threads=\threads, count=\count, average=\average, median=\median, quantil=\quantil, min=\min, max=\max, err=\err, rate=\rate, personalized=\personalized, cores=\cores }%
{ \threads & \rate & \min & \max & \median & \quantil & \err & \count }%
}

\newpage
\section{Referenzierte Repositories}
\label{app:repos}

Übersicht der Repositories für die im Rahmen dieser Arbeit erstellten Softwaremodule und Konfigurationsartefakte.

\begin{itemize}
\item \textbf{``Recommend''} - Hauptrepository mit allen Integrationskomponenten für beide Personalisierungslösungen und allen in Abschnitt \ref{sec:mahoutext} beschriebenen Ergänzungen für Apache Mahout. \\ https://gitorious.org/recommend/recommend
\item \textbf{``Tracker''} -  Umsetzung des in Abschnitt \ref{sec:tracker-impl} beschriebenen Protokollierungsdienstes. \\ https://gitorious.org/recommend/tracker
\item \textbf{``IMDB Index App``} - Beispielanwendung welche, basierend of NodeJS, beide Möglichkeiten der Personalisierung gegenüberstellt. \\ https://gitorious.org/recommend/imdb-index
\item \textbf{``Datenaufbereitung''}  - Scripte zur Datenaufbereitung für die Modellberechnung (vgl. Abschnitt \ref{sec:dataprep_real}). \\ https://gitorious.org/recommend/pig-preparation
\item \textbf{``Performance Tests''} - Apache Jmeter Konfigurationsdateien für die in Abschnitt \ref{sec:performance} durchgeführten Leistungsevaluationen. \\ https://gitorious.org/recommend/performance-tests
%\item recommend/pig-udfs
\item \textbf{``Apache Solr Konfiguration''} - Beispielkonfigurationen für die im Rahmen der Arbeit genutzten Apache Solr Instanzen. \\ https://gitorious.org/recommend/solr-config
\end{itemize}


\end{appendices}

%\clearpage
%\addcontentsline{toc}{chapter}{Index}

%\printindex

\end{document}
